\begin{frame}{Introduction}
    \begin{itemize}
        \item Validate ALMA9 version of Calypso for the track variables
        \item Sinead already looked at single muon samples
        \item Ansh looked at the A' analysis cutflow
        % \item For the Dark-Photon Samples
        \item We are looking at the ``two track reconstruction'' as a function of separation between them
    \end{itemize}
\end{frame}

\begin{frame}{Data Description}
    \begin{itemize}
        \item We analysed the Dark-Photon decay to electron pairs
        \item Data samples used are \href{/eos/experiment/faser/data0/sim/mc24/foresee/1100*/phy/}{\small /eos/experiment/faser/data0/sim/mc24/foresee/1100\{33,38,51\}/}
        \begin{itemize}
            \item 110033 : Mass = 10  MeV, epsilon = 1E-5 
            \item 110038 : Mass = 100 MeV, epsilon = 1E-5 
            \item 110051 : Mass = 10  MeV, epsilon = 1E-4 
        \end{itemize}
        \item ALMA 9 samples : \href{./phy/s0008-dev/}{./phy/s0008-dev/} 
        \item CENTOS 7 samples: \href{./phy/s0008-r0019/}{./phy/s0008-r0019/}
        \item Chaining them together  for better statistics [total 60k events]
        \item Can separate based on mass/couplings if interested
    \end{itemize}
    
\end{frame}

\begin{frame}{Overview of Validation }
    \textbf{Objective: Quantify the Efficiency of  two track reconstruction  as a function of separation between tracks}
    \begin{itemize}
        \item Perform an initial assessment of the Track Parameters
        % \begin{itemize}
        %     \item longTracks
        %     \item Track Chi2 / Track Chi2/DoF
        %     \item Track nDoF
        %     \item Track Charge
        %     \item \ldots
        % \end{itemize}
        \item Quantify the separation between tracks
        \item Evaluate generic track reconstruction performance
        \item Define a metric for ``Reconstruction Efficiency'' 
        % \item Analyze residuals to evaluate accuracy

        
    \end{itemize}

\end{frame}