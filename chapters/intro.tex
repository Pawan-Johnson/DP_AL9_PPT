\begin{frame}{Introduction}
    \begin{itemize}
        \item Validate Alma9 version of Calypso for the track variables
        \item For the Dark-Photon Samples
        \item Specifically to understand the difference in two track reconstruction when they are close by
    \end{itemize}
\end{frame}

\begin{frame}{Data Description}
    \begin{itemize}
        \item We want to look at Dark Photon decays to two tracks.
        \item Data samples used are \href{/eos/experiment/faser/data0/sim/mc24/foresee/1100*/phy/}{/eos/experiment/faser/data0/sim/mc24/foresee/1100\{33,38,51\}/}
        \vspace{-0.25cm}\begin{itemize}
            \item 110033 : Mass = 10  MeV, epsilon = 1E-5 
            \item 110038 : Mass = 100 MeV, epsilon = 1E-5 
            \item 110051 : Mass = 10  MeV, epsilon = 1E-4 
        \end{itemize}
        \item ALMA 9 samples : \href{./phy/s0008-dev/}{./phy/s0008-dev/} 
        \item CENTOS 7 samples: \href{./phy/s0008-r0019/}{./phy/s0008-r0019/}
        \item Chaning them together gives a total of 60k events.
        \item {\color{red}{Justification for chaining given diff mass/couplings?}}
    \end{itemize}
    
\end{frame}

\begin{frame}{Overview of Validation }
    \begin{itemize}
        \item Begin with a sanity check by looking at the TrackParameters
        \begin{itemize}
            \item longTracks
            \item Track Chi2 / Track Chi2/DoF
            \item Track nDoF
            \item Track Charge
            \item \ldots
        \end{itemize}
        \item Quantify separation between tracks
        \item Compare track reconstruction as a function of above
        \item Definition of Efficiency?
        \item Residues? 

        
    \end{itemize}

\end{frame}